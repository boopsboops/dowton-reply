\documentclass[a4paper,12pt]{letter}
\usepackage{graphicx}
\usepackage[T1]{fontenc}% change to nicer font
\usepackage[bitstream-charter]{mathdesign}% change to nicer font
%\usepackage[a4paper, top=4cm, bottom=2.5cm, left=4cm, right=4cm]{geometry}
\usepackage{pdfpages}
\usepackage{url}
\pagestyle{empty}

%\name{}
\signature{Rupert A.\ Collins \& Robert H.\ Cruickshank} 
\address{Laborat\'orio de Evolu\c c\~ao e Gen\'etica Animal \\ Departamento de Biologia \\ Universidade Federal do Amazonas \\ Manaus, Amazonas, Brasil \\ Email: \normalsize{\textsf{rupertcollins@gmail.com}}}

\begin{document}
\begin{letter}{Prof.\ Frank Anderson \\ Editor-in-chief, \emph{Systematic Biology}}

\opening{Dear Prof.\ Anderson,} 

Please find attached the manuscript entitled ``Known knowns, known unknowns, unknown unknowns and unknown knowns in DNA barcoding: a comment on Dowton et al.'' by Rupert A.\ Collins and Robert H.\ Cruickshank for consideration as a point-of-view article in \emph{Systematic Biology}. 

Our manuscript presents a critique of a study currently in-press at \emph{Systematic Biology}: ``A preliminary framework for DNA barcoding, incorporating the multispecies coalescent'' by Dowton et al.\ ({\small \url{http://dx.doi.org/10.1093/sysbio/syu028}}). While we appreciate the aims of their study, we demonstrate that their conclusions misrepresent their data, and do not provide a sufficiently compelling argument that their mitochondrial COI data are insufficiently sensitive to resolve the identification problems presented. Disregarding the appropriateness of the dataset they chose to use, we feel that this also represents an ideal opportunity to explore some of the more general issues raised in their study, and other recent opinion pieces (e.g.\ Taylor \& Harris, 2012 \emph{Molecular Ecology}). Particularly, we discuss whether such an endeavour such as single locus DNA barcoding will play a future role in an increasingly genomics-dominated landscape.

We hope that you agree with us that this discussion will help generate more rigorous investigations into the situations in which DNA barcoding succeeds and fails, and also how the discipline will move forward and embrace post-Sanger sequencing technologies. Our manuscript is original work and not under consideration at any other journal. We recommend Tanja Stadler as Academic Editor, as she handled the original manuscript of Dowton et al.


\closing{Yours faithfully,\\ 
\vspace{10 mm}
\includegraphics[width=2in]{signature.png}}

%\newpage
%\includepdf[pages=-]{reviewer_response.pdf}

\end{letter}
\end{document}
% Mark de Bruyn <markus.debruyn@gmail.com> Bangor University, UK
% Ulrich Schliewen <schliewen@zsm.mwn.de> ZSM, Germany
% Prosanta Chakrabarty <prosanta@lsu.edu> Louisiana State University, USA
